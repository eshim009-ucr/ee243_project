% Full quotes of citations to show how the primary work has been referenced
\begin{itemize}
	\item ``In this work, a relatively small dataset, MNIST, has been used as a
	test vehicle to validate the proposed method to accommodate the limited
	computation resources. Table 2 [which includes \citeauthor{main}] presents
	different network configurations with their corresponding parameter
	precision, data format, and memory utilisation''
	\autocite[9]{phipps2023pre}.
	\item ``{\ldots}the NetMnist handwritten digit classifier and the Residual
	Network (Resnet) architectures were tested using the 2-d images of each
	breath''
	\autocite[2]{baedorf2023reverse}.
	\item ``As a result of their high performance, energy efficiency, and
	reconfigurability, FPGA-based implementations have attracted considerable
	interest''
	\autocite[2]{aydin2023fpga}.
	\item ``\citeauthor{main} also develop an accelerator on FPGA, optimizing
	the design using resource multiplexing and parallel processing, limiting the
	implementation to 3×3 kernels, to avoid issues with reconfiguration''
	\autocite[5]{pistellato2023quantization}.
	\item ``While the LUTs in \citeauthor{main} are less than half when compared
	to our proposed work, \citeauthor{main} uses resource multiplexing, wherein
	the modules are reused for different operations cycles, which tends to
	higher latency. In \citeauthor{main}, as an example, digit recognition
	requires 68,139 clock cycles{\ldots}our proposed 32-bit design achieves a
	higher accuracy than that of the implementations with lower precision,
	including the{\ldots}16-bit fixed design in \citeauthor{main}\ldots''
	\autocite[16]{madineni2023parameterizable}.
	\item ``{\ldots}because convolutional neural networks entail several
	calculations and the optimization of numerous matrices, their application
	necessitates the utilization of appropriate technology, such as GPUs or
	other accelerators''
	\autocite[241]{naufal2023comparative}.
	\item ``After much effort, this problem has been alleviated through
	techniques such as{\ldots}data augmentation''
	\autocite[2]{yan2023end}.
\end{itemize}
