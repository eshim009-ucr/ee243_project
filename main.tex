\documentclass[12pt,letterpaper,english]{article}

% Packages
\usepackage[utf8]{inputenc}
\usepackage{amssymb}
% For colored text
\usepackage{xcolor}
\definecolor{gren}{rgb}{0, 0.6, 0}
\definecolor{blu}{rgb}{0, 0, 0.6}
% For SI unit notation
\usepackage{siunitx}
% For citations
\usepackage[backend=biber,sorting=none]{biblatex}
% For language and localization
\usepackage{babel}


% Macros
% Mark placeholder fields
\newcommand{\todo}[1][TODO]{{\color{red}#1}}

\hypersetup{
	pdftitle = {%
		Decreasing Power Usage of
		Convolutional Neural Network Implementations
		on FPGAs
	},
	pdfauthor = {Clarity Shimoniak},
	pdfsubject = {EE 243 Project Proposal},
	pdfkeywords = {%
		computer vision, convolutional neural network, CNN, FPGA, Xilinx,
		Lattice, MNIST
	},
}


\addbibresource{models.bib}
\addbibresource{fpga.bib}
\addbibresource{related.bib}
\newcommand{\citework}[1]{\citeauthor{#1} \autocite{#1}}
\newcommand{\citeworks}[2]{%
	\citeauthor*{#1} \autocite{#1} and \citeauthor*{#2} \autocite{#2}%
}
\newcommand{\citemoreworks}[3]{%
	\citeauthor*{#1} \autocite{#1},
	\citeauthor*{#2} \autocite{#2}, and
	\citeauthor*{#3} \autocite{#3}%
}


\begin{document}

\begin{center}
	{\LARGE%
		Decreasing Power Usage of
		Convolutional Neural Network Implementations
		on FPGAs
	} \\
	\vspace{6pt}
	Clarity Shimoniak
	\vspace{-18pt}
\end{center}

\section*{Key Publication}

In \citetitle{mobilenet2019fpga} \autocite{mobilenet2019fpga},
\citeauthor{mobilenet2019fpga} implement the MobileNet model
\autocites{mobilenetv1}{mobilenetv2} on a Xilinx MPSoC board which integrates an
FPGA and an ARM processor. They claim a more efficient FPGA implementation due
to the use of specialized engines for standard and depthwise convolution as
opposed to using the same engine for both.


\section*{Topic Introduction}

The high power usage of most CNN implementations limits their viability for
battery-powered applications. Additionally, large and expensive acceleration
hardware like GPUs limits their viability for consumer applications. Increasing
power efficiency and decreasing component cost would allow these models to be
deployed far more widely.


\section*{Planned Activities}

I plan to implement a system similar to \citeauthor*{mobilenet2019fpga}'s using
the Lattice iCE40UP5k FPGA instead of the heterogeneous Xilinx SoC. The focus
will be on reducing power usage while maintaining a reasonable standard of speed
and quality.

If time permits, I will also implement the system on other Lattice FPGAs of
slightly higher and lower specs (LFE5U-25F-6 and iCE40LP1k respectively).
Achieving similar results without using an SoC would result in an implementation
that is suited to more resource- and cost-constrained embedded applications.


\section*{Related Publications}

\begin{itemize}
	\item \citetitle{synetgy} \autocite{synetgy} uses a modified
	ShuffleNetv2 \autocite{shufflenetv2} architecture optimized for FPGAs.
	\item \citework{mobilenet2021fpga} use a Xilinx Virtex 7 FPGA to create a
	high-performance MobileNet implementation capable of over $\SI{300}{FPS}$
	and offering a speedup over various CPU and GPU implementations.
	\item \citework{mobilenet2018fpga} use an Altera Arria 10 heterogeneous SoC
	to create a high-performance MobileNetv2 implementation capable of over
	$\SI{250}{FPS}$.
	\item \citework{mobilenet2018rr} introduce RR-MobileNet, a variant of
	MobileNetv1 that achieves over $\SI{120}{FPS}$ on a Xilinx heterogeneous
	UltraScale MPSoC.
\end{itemize}


\newpage
\printbibliography


\end{document}
