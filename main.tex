\documentclass[12pt,letterpaper,english]{article}

% Packages
\usepackage[utf8]{inputenc}
\usepackage{amssymb}
% For colored text
\usepackage{xcolor}
\definecolor{gren}{rgb}{0, 0.6, 0}
\definecolor{blu}{rgb}{0, 0, 0.6}
% For SI unit notation
\usepackage{siunitx}
% For citations
\usepackage[backend=biber,sorting=none]{biblatex}
% For language and localization
\usepackage{babel}


% Macros
% Mark placeholder fields
\newcommand{\todo}[1][TODO]{{\color{red}#1}}

\hypersetup{
	pdftitle = {EE 243 Project Proposal},
	pdfauthor = {Clarity Shimoniak},
%	pdfsubject = {},
%	pdfkeywords = {},
}


\addbibresource{cites.bib}
\addbibresource{cites-to-main.bib}
\addbibresource{cites-from-main.bib}


\begin{document}

\begin{center}
	{\LARGE%
		Decreasing Power Usage of
		Convolutional Neural Networks Implementations
		on FPGAs
	} \\
	\vspace{6pt}
	Clarity Shimoniak
\end{center}

\section*{Key Publication}

In \citetitle{main}, \citeauthor{main} implement handwritten digit recognition
using the MNIST dataset on a Xilinx Artix 7 FPGA. Their primary goal is to
reduce the usage of hardware resources and power consumption of the system while
maintaining acceptable performance.


\section*{Topic Introduction}

\todo[A concise introduction to the topic.]


\section*{Related Publications}

\begin{itemize}
	\item \citeauthor{madineni2023parameterizable} propose a system implemented
	in the new Chisel hardware design language that claims improvements over
	\citetitle{main} and several similar implementation.
	\item \citeauthor{baedorf2023reverse} reference \citetitle{main} as a method
	of classification that was passed over for neural network analysis of
	abnormal breathing.
	\item \citeauthor{phipps2023pre} contrast the performance of
	\citetitle{main} against other similar models.
	\item \citeauthor*{pistellato2023quantization} and \citeauthor*{yan2023end}
	use \citetitle{main} as an example of techniques by which performance of
	neural networks on FPGAs can be improved.
	\item \citeauthor*{aydin2023fpga} and \citeauthor*{naufal2023comparative}
	use \citetitle{main} as a generic example of the significance of hardware
	acceleration for neural networks.
\end{itemize}


\section*{Planned Activities}

\todo[Description of your planned activities.]


\newpage
\printbibliography


\end{document}
